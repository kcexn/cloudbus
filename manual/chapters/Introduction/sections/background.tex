\section{Background}
Middleware frameworks are often monolithic and have been designed to suit the purposes of a single organization, examples include Kubernetes %
YARN, and Mesos. In most designs, application scheduling and resource management is done independently of networked communication. This limits %
the ability for the resource management daemons in these frameworks to use feedback mechanisms to improve application performance via enhanced %
resource managmement. 

Contemporary research in queuing and information theory indicates that job schedules can be improved by integrating the following %
ideas into resource management frameworks:
\begin{itemize}
	\item Time-sharing application execution over parallel and statistically independent compute resources. This is best achieved by %
	breaking application execution into many small atomic functions that can be executed idempotently on any node in the network.
	\item Take full advantage of idle network capacity to spread workloads more evenly across the cluster via distributed state-sharing %
	--- this improves application performance even when taking the communication latency between nodes into consideration.
	\item Take full advantage of idle compute resources via opportunistic CPU and memory bursting. This increases the probability of resource %
	starvation at the per-node level but this risk can be mitigated by replicating workloads across multiple independent nodes.
	\item Emphasizing the mitigation of long-tail events that degrade the aggregate performance of all applications that are multiplexed %
	on a given cluster. This means ensuring that the steady-state probability of a long-tail event should converge to 0. Long-tail %
	events should be considered any event whose performance measurements indicate that the hazard-rate (halting-rate) function will %
	be decreasing (less likely to stop with more time) rather than non-decreasing.
\end{itemize}

Cloudbus strives to be compatible with existing resource management frameworks for cloud computing and large-scale distributed data processing.