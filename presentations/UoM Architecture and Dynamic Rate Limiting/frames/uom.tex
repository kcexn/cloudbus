\begin{frame}
	\frametitle{UoM Student Engagement --- Opportunities}
	\only<1>{
		\begin{block}{Transport protocols}
			Cloudbus has a working implementation of TCP, and unix domain sockets. To be transport agnostic, Cloudbus needs to support:
			\begin{itemize}
				\item UDP,
				\item SCTP,
				\item and abstract unix domain sockets.
			\end{itemize}
		\end{block}
	}
	\only<2>{
		\begin{block}{HTTP gateway}
			Cloudbus has no way for users to authenticate, nor for users to make dynamic resource requests. An RFC compliant %
			implementation of HTTP/1.1 is planned to provide these capabilities:
			\small
			\begin{itemize}
				\item Deserialize/Serialize HTTP/1.1 requests from/to cloudbus data frames.
				\item Forward the user payload based on the header and payload information in the HTTP request.
				\item Bind multiple HTTP requests to a single cloudbus session.
				\item Websocket support.
			\end{itemize}
			\normalsize
		\end{block}
	}
	\only<3>{
		\begin{block}{Cloudbus configuration and systemd service units}
			For cloudbus components to be successfully installed onto a live system. The software needs:
			\small
			\begin{itemize}
				\item Proper configuration files i.e., ini files.
				\item Installable systemd service units.
				\item Configurable Makefiles for target specific builds.
				\item Bundling for package managers (apt, zypper, yum, pacman, etc.)
			\end{itemize}
			\normalsize
		\end{block}	
	}
	\only<4>{
		\begin{block}{Dynamic DNS Design and Configuration}
			Dynamic DNS Zones can be used to keep a distributed record of services that cloudbus can access. DNS supports several records that %
			can be used to expose services onto a network using arbitrary names:
			\begin{itemize}
				\item SRV records,
				\item SVCB records,
				\item NAPTR records.
			\end{itemize}
		\end{block}
	}
	\only<5>{
		\begin{center}
			\LARGE \href{https://github.com/kcexn/cloudbus}{https://github.com/kcexn/cloudbus}
		\end{center}
	}
\end{frame}