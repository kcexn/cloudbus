\begin{frame}
	\frametitle{UoM Student Engagement --- Opportunities}
	\only<1>{
		\begin{block}{Open-source and AGPL licensed.}
			Cloudbus is open-source and Affero GPL licensed. UoM Students:\vspace{1em}
			\small
			\begin{itemize}
				\item Will retain the copyright for any proposed merge requests to the cloudbus project\footnote{subject to university conditions}.
				\item Will be listed as a \emph{contributor} for approved merge requests.
				\item Depending on size and complexity, will also be listed as a \emph{maintainer} for approved merge requests.
			\end{itemize}
			\normalsize
		\end{block}
	}
	\only<2>{
		\begin{block}{RFC Compliant UUIDs}
			Universally unique identifiers have many useful features that can be used for session state-management. An RFC compliant implementation %
			of UUIDs will allow Cloudbus to take full advantage of these features. Cloudbus still needs the following UUID versions: 
			\small
			\begin{itemize}
				\item V1, V2, V3, V5, V6
			\end{itemize}
			\normalsize
		\end{block}		
	}
	\only<3>{
		\begin{block}{Transport protocols}
			Cloudbus has a working implementation of TCP, and unix domain sockets. To be transport agnostic, Cloudbus needs to support:
			\small
			\begin{itemize}
				\item UDP,
				\item SCTP,
				\item and abstract unix domain sockets.
			\end{itemize}
			\normalsize
		\end{block}
	}
	\only<4>{
		\begin{block}{Cloudbus configuration and systemd service units}
			For cloudbus components to be successfully installed onto a live system. The software needs:
			\small
			\begin{itemize}
				\item Proper configuration files i.e., ini files.
				\item Installable systemd service units.
				\item Configurable Makefiles for target specific builds.
				\item Bundling for package managers (apt, zypper, yum, pacman, etc.)
			\end{itemize}
			\normalsize
		\end{block}	
	}	
	\only<5>{
		\begin{block}{HTTP gateway}
			Cloudbus has no way for users to authenticate, nor for users to make dynamic resource requests. An RFC compliant %
			implementation of HTTP/1.1 is planned to provide these capabilities:
			\small
			\begin{itemize}
				\item Deserialize/serialize HTTP/1.1 requests from/to cloudbus data frames.
				\item Forward user payloads based on HTTP header and payload information.
				\item Bind multiple HTTP requests to a single cloudbus session.
				\item Websocket support.
			\end{itemize}
			\normalsize
		\end{block}
	}
	\only<6>{
		\begin{block}{Dynamic DNS design and implementation}
			Dynamic DNS zones can be used to keep a distributed record of services that cloudbus can access. DNS supports several records that %
			can be used to expose named services onto a network:
			\small
			\begin{itemize}
				\item SRV records,
				\item SVCB records,
				\item NAPTR records.
			\end{itemize}
			\normalsize
			Cloudbus needs a strategy for resolving DNS queries, it needs a convention for naming distributed services, %
			and it needs a DNS client that implements these strategies and can autonomously update a dynamic DNS zone.
		\end{block}
	}
	\only<6>{
		\begin{center}
			\LARGE \href{https://github.com/kcexn/cloudbus}{https://github.com/kcexn/cloudbus}
		\end{center}
	}
\end{frame}